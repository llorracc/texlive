% Obtained by CDC on 5/23/05 from http://cvs.sourceforge.net/viewcvs.py/prosper/prosper/TODO?rev=1.6

%  Joe B. Wells has provided the following patch which determines the
%  number of overlays needed from the \fromSlide, \onlySlide, and
%  \untilSlide on the slide:

    \newcounter{currentOverlay}
    \newcounter{maxOverlay}
    
    % \recordOverlay{n} tells \overlayslide that overlay n exists.
    
    \newcommand{\recordOverlay}[1]
      {\ifnum #1>\value{maxOverlay}\relax
         \setcounter{maxOverlay}{#1}%
       \fi}
    
    % All material after \step appears on the next and subsequent
    % overlays.  This also tells \overlayslide that the overlays exist.
    
    \newcommand{\step}
      {\StepCounter{currentOverlay}%
       \recordOverlay{\value{currentOverlay}}%
       \FromSlide{\thecurrentOverlay}}
    
    % This command should probably not exist.  Don't use it.  Just write
    % \step\item where needed.
    
    \newcommand{\stepitem}{\step\item}
    
    % All material after \FromOverlay{n} (until the next such command)
    % will appear on overlay n and all later overlays.  This also tells
    % \overlayslide that overlay n exists.
    
    \newcommand{\FromOverlay}[1]{\recordOverlay{#1}\FromSlide{#1}}
    
    % \overlayslide{XYZ}{...} is like doing
    % \overlays{n}{\begin{slide}{XYZ}...\end{slide}} except that you can
    % also use the \step and \FromOverlay commands defined just above and
    % the value of n will be automatically determined from the uses of
    % \step and \FromOverlay.
    
    \long\def\overlayslide#1#2%
      {% #1 is TITLE
       % #2 is BODY
       \global\InOverlaystrue % we already _are_ global!!!
       \aftergroup\InOverlaysfalse % this seems completely daft!
       \setcounter{overlaysCount}{1}%
       \setcounter{maxOverlay}{1}%
       \ifDVItoPS
         \setcounter{currentOverlay}{1}
         \begin{slide}{#1}
           #2%
           \par\hbox{}% bizarre hack which works around strange Prosper problem
         \end{slide}
       \else
         \begin{Overlays}
         \bgroup
           \loop 
             \setcounter{currentOverlay}{1}%
             \begin{slide}{#1}%
               #2%
               \par\hbox{}% bizarre hack which works around strange Prosper problem
             \end{slide}%
           \ifnum\value{overlaysCount}<\value{maxOverlay}\relax
             \StepCounter{overlaysCount}%
           \repeat
         \egroup
         \end{Overlays}
       \fi}


