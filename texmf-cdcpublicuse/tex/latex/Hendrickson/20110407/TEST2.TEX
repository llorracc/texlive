\documentclass{book}
\advance\textheight1in

\usepackage{printvrb}
\input test.vrb

\namesinmargin
\begin{document}
\parskip=18pt


Here we have our first test fragment:\\

\printfrag{EulerIsMyFriend}


\begin{itemize}
\item
Here we print the second frag:\\ 

\printfrag{SomethingElse}

\item
Here's the third test with a number in it:

\printfrag{ThirdTest223}

\end{itemize}

And Here is an example of using a fragment more
than once. 

\printfrag{EulerIsMyFriend}

If you want to set the counter for the equation numbers
you can do that too, with \verb+\setcounter{equation}{0}+,
and the same equations numbers will be used.

\setcounter{equation}{0}

\printfrag{EulerIsMyFriend}

====

Now we see what happens if you call for a name that isn't
defined:

\printfrag{Mistake}

\end{document}
